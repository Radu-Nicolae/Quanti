\documentclass[sigconf,review]{acmart}
\usepackage{amssymb}
\usepackage{pifont} % for \cmark and \xmark
\usepackage{pifont} % for \cmark and \xmark
\newcommand{\cmark}{\ding{51}} % check mark
\newcommand{\xmark}{\ding{55}} % cross mark

\usepackage{amsmath,amssymb,amsfonts,latexsym}
\usepackage{enumerate}
\usepackage{lipsum}
\usepackage{xspace}
\usepackage{epsf,picinpar}
\usepackage{hyperref}
\usepackage{varioref}
\usepackage{varioref}
\usepackage{tabularx}
\usepackage{colortbl,multirow,hhline}
\usepackage{listings}
\usepackage{amssymb}
\usepackage{colortbl,multirow,hhline}
\usepackage{algorithmic}
\usepackage{algorithm}
\usepackage{caption}
\usepackage[normalem]{ulem}
\usepackage{cleveref}
\usepackage{xcolor}
\usepackage{pifont}
\usepackage{xcolor,colortbl}
\usepackage{url}
\usepackage{balance}
\usepackage{graphicx, subfigure}
\usepackage{longtable}
\usepackage{lscape}
\usepackage{multirow}
\usepackage{listings}
\usepackage{framed}
\usepackage{morefloats}
\usepackage[T1]{fontenc}
\usepackage{array}
\usepackage{pdfpages}
\usepackage{fancybox}
\usepackage{amsmath}
\usepackage{flushend}
\usepackage{booktabs}
\usepackage{enumitem}

\renewcommand{\ttdefault}{cmr}

%\newcommand{\limit}[1]{\textcolor{red}{\noindent \ding{46}~Page limit:~#1}\\}
\newcommand{\todo}[1]{\textcolor{blue}{\ding{46}~#1}} 
\newcommand{\ie}{\emph{i.e.,}\xspace}
\newcommand{\eg}{\emph{e.g.,}\xspace}
\newcommand{\etc}{etc.\xspace}
\newcommand{\etal}{\emph{et~al.}\xspace} 

\copyrightyear{2025}
\acmYear{2025}
\setcopyright{acmcopyright}
\acmConference[Green Lab 2025]{Green Lab 2025/2026 - Vrije Universiteit Amsterdam}
\acmBooktitle{Vrije Universiteit Amsterdam}
\acmPrice{}
\acmDOI{}
\acmISBN{}

    
\begin{document}


\title{
Exploring the impact of LLM compression by quantization on energy consumption, resource utilisation, and accuracy
}

\author{Radu Nicolae}
\affiliation{%
 \institution{2760443 \\ VU Amsterdam, \\ Universiteit van Amsterdam}
} \email{r.nicolae@vu.nl}

\author{Junaid Khalil}
\affiliation{%
 \institution{2891638 \\ VU Amsterdam, \\
 Universiteit van Amsterdam}
} \email{j.khalil6@student.vu.nl}

\author{Muhammad Shayan}
\affiliation{%
 \institution{2891482 \\ VU Amsterdam}
} \email{m.shayan@student.vu.nl}

\author{Md Tasluf Morshed}
\affiliation{%
 \institution{2890836 \\ VU Amsterdam}
} \email{m.t.morshed@student.vu.nl}

\author{Alp Eren Inceoglu}
\affiliation{%
 \institution{2842116 \\ VU Amsterdam}
} \email{a.e.inceoglu@student.vu.nl}

\begin{abstract}
Large Language Models (LLMs) are being increasingly adopted by our digital society, but raise sustainability concerns when operated at a massive societal scale. % societal context
To improve the performance, efficiency, and, thus, sustainability of LLMs, compression techniques have been widely adopted, yet at the cost of accuracy. % LLM context
One such technique is quantization, a state-of-the-art approach to reduce model size (thus computation requirements), while maintaining sufficient accuracy, and without architectural changes or re-training. % quantization context
However, although quantization is theoretically expected to generate tradeoffs between energy efficiency, resource utilization (e.g., CPU, memory, inference time), and accuracy degradation, these tradeoffs are insufficiently understood and unquantified. % gap 1
The quantification challenge is further exacerbated by the absence of benchmarking systems for systematically quantifying LLM ecosystems; the lack of such systems can be costly and could misguide operators of LLM services. % gap 2
Addressing the open challenge, in this work, we propose \underline{Quanti}, the first tool for \underline{quanti}fying the sustainability and performance of LLMs. % approach (following AtLarge, community-standard methodology on designing distributed (eco)systems)
We propose an architecture for deploying, measuring, and comparing LLM on the server side. % design
Through experiments with a prototype, % implement
we % evaluate & experiment
(i) explore the impact of quantization on operational-level metrics of LLMs,
(ii) evaluate the tradeoff between energy consumption-accuracy of LLMs underlying different architectures, both original and under compression by quantization, and
(iii) quantify the impact of CO2-aware workload scheduling of compressed and uncompressed models. 
Quanti, together with all the production data, results, and reproducibility capsule, are released open-source on \url{https://github.com/Radu-Nicolae/Quanti}.

\vspace*{-0.1cm}



% \noindent \textit{Goal}. 
% \todo{at the end}

% \noindent \textit{Method}. 
% \todo{at the end}

% \noindent \textit{Results}. 
% \todo{at the end}

% \noindent \textit{Conclusions}. 
% \todo{at the end}
\end{abstract}


\maketitle

\section{Introduction}


\begin{figure}[H]
    \centering
    \includegraphics[width=0.95\linewidth]{reportTemplate/figures/f1.pdf}
    \caption{Mock! \underline{Quanti}fying the accuracy of LLMs, and energy consumption of LLMs under workload, compressed and uncompressed (baseline model) by quantization. Expanded analysis in §add section here.}
    \label{fig:f1}
    
\end{figure}
% \lipsum[1-7]
Large Language Models (LLMs) have been used in a large number of applications of artificial intelligence recently, particularly in natural language processing (NLP) and software engineering. \cite{DBLP:journals/corr/abs-1910-01108} \cite{10968787}. However, these models are often so large that they contain billions of parameters ultimately creating significant sustainability challenges when used on a massive scale. \cite{DBLP:journals/tacl/ZhuLLMW24} \cite{DBLP:journals/corr/abs-1910-01108}. The trend of using Large Language Models in applications is on the rise, resulting in exponentially scaling computational requirements and substantial environmental costs. For example, one research shows that 16.68 tons of CO2 are emitted just by training a model called StarCoder, with 7 billion parameters. \cite{DBLP:journals/corr/abs-2507-09665}. While training is a considerable expense, a large number of environmental and computational costs typically rise after deployment, with inference energy consumption often bypassing training over time \cite{DBLP:journals/corr/abs-2507-09665}. These rising demands also put substantial memory and processing burdens, which slow down the widespread adoption of Large Language models in real-time applications \cite{DBLP:journals/corr/abs-1910-01108}.



To tackle these important challenges, researchers have explored several compression techniques to improve the performance, efficiency, and overall sustainability of LLMs. Although it is important to make LLMs more manageable, LLM comparison can introduce inherent tradeoffs, most notably a potential reduction in accuracy because of such compression methods. Among these is a compression technique known as quantization, which is used to decrease the parameter precision (e.g., from 32-bit floating point to 8-bit or 4-bit integers), which reduces the model size and computational demands without affecting the performance significantly or requiring any extensive retraining \cite{DBLP:journals/corr/abs-2507-09665} \cite{10968787}. To reduce the demands for LLM resources, the CO2 footprint and energy consumption, quantization has been chosen as a reliable choice \cite{DBLP:journals/corr/abs-2507-09665}. Empirical evidence shows that quantization can lead to reductions in energy consumption and carbon emissions up to 45\% after quantification \cite{DBLP:journals/corr/abs-2504-06307}. It can also reduce the memory utilization by 50\% (e.g., from 500 MB to 250 MB) and improve the inference time (from 150 to 110 ms), which maintains high precision (e.g., 91. 2\%)  \cite{10968787}. 

%Other established compression strategies, such as knowledge distillation and pruning, also offer substantial benefits. Knowledge distillation, for example, produced DistilBERT, which is 40\% smaller and 60\% faster than BERT while retaining 97\% of its language understanding capabilities, making it cheaper to pre-train. Distilled models generally consume less energy, exhibit faster inference times, and, in some cases, reduce CPU and memory usage, with Distilled-BERT achieving a 43.96\% reduction in energy consumption. Similarly, pruning techniques can achieve a reduction of up to 60\% in model size, significantly improving inference times. The choice among these techniques depends on specific application requirements and acceptable trade-offs.

Regardless of the development made so far, systematic evaluations of the trade-offs introduced by quantization in LLMs are still lacking. Current benchmarking tools are often too limited to capture the complexity of LLM ecosystems, which can lead service providers to make a poorly informed or costly deployment decision\cite{10968787}.  Although some recent work has examined code-focused LLMs, comprehensive studies addressing energy consumption, carbon intensity, and overall operational performance remain scarce. Without reliable and thorough measurement frameworks, stakeholders risk relying on inefficient or unsustainable approaches.

To help address this need, we present Quanti, a tool developed to assess both the sustainability and performance of quantized LLMs. Quanti supports the systematic deployment, measurement, and comparison of models, with particular emphasis on server-side environments. Quanti enables stakeholders from various groups (e.g., LLM operators, researchers, students) to conveniently evaluate the sustainability and performance of LLMs. For example, Quanti can assist LLM operators in assessing the total energy consumption and accuracy of various quantization methods. It can also support students in comparing the accuracy of different architectural LLMs under compression by quantization.

This study examines the broader effects of quantization on energy use, resource demands, and model accuracy, guided by the main research question: How to explore the impact of compression by quantization of LLMs and how this technique impacts energy consumption, resource utilization, and precision?
% \begin{itemize}
% \item \textbf{MRQ:} How to explore the impact of compression by quantization of LLMs and how this technique impacts energy consumption, resource utilization, and precision

% \item \textbf{RQ1:} How to design and implement a benchmarking system for quantifying LLMs?
% \item \textbf{RQ2:} How to evaluate the impact of compression by quantization of operational-level metrics of LLMs?

% \item \textbf{RQ3:} How to evaluate the tradeoff accuracy-energy consumption across various-architecture LLMs under quantization and without quantization?

% \item \textbf{RQ4:} How to evaluate the impact of carbon intensity for compressed and uncompressed models at favourable and unfavourable carbon intensity day instances?


% \end{itemize}

To explore these questions, our experimental approach uses quantization as the primary compression method for LLMs. We will apply this technique to at least three models and their quantized versions that can be run locally, including Granite-7B, LLaMA-2 7B, and Mistral-7B. Each model will be evaluated with a consistent benchmarking setup to compare energy use, CPU and memory demands, inference latency, and accuracy against its uncompressed baseline. The results of this work are intended to benefit several key stakeholders, namely LLM operators, researchers, and students. 


% \textbf{MRQ:} How to explore the impact of compression by quantization of LLMs and how this technique impacts energy consumption, resource utilization, and accuracy?

% \textbf{RQ1:} How to design and implement a benchmarking system for quantifying LLMs?

% \textbf{RQ2:} How to evaluate the impact of compression by quantization of operational-level metrics of LLMs?

% \textbf{RQ3:} How to evaluate the tradeoff accuracy-energy consumption across various-architecture LLMs under quantization and without qunatization?

% \textbf{RQ4:} How to evaluate the impact of carbon intensity compressed and uncompressed models at favourable and unfavourable carbon intensity day instances?

% Carbon intesity 
% - night is the worst
% - day is the best



% This document represents a template of the final experiment report structure for the course \textit{Green Lab} at the Vrije Universiteit Amsterdam \cite{greenlab}.

% The experiment is conducted according to the guidelines by Wohlin and colleagues \cite{wohlin12, DBLP:books/sp/WohlinRHORW24}.

% The total length of this document must not exceed 15 pages, including references, appendixes, \etc

% In this section you have to describe 
% (i) the domain (\eg mobile apps and their market) and the technologies relevant for understanding the rest of the document, 


% (ii) the main motivation behind your experiment (the problem, here you can show examples via apps/tools screenshots, snippets of source code, \etc), 

% (iii) what your experiment is about (hint of the solution), and 

% (iv) what the developers will learn from the results of your experiment.  

% \textcolor{red}{Page limit: 2}
 \newpage
\section{Background: quantization and quantification} \label{sec:background}

\begin{figure}
    \centering
    \includegraphics[width=0.75\linewidth]{reportTemplate/figures/f2.png}
    \caption{Caption}
    \label{fig:background}
\end{figure}

In this section, we present background on LLM quantization and a high-level overview of this compression technique in~\Cref{fig:background}. Then, we present background on metrics and techniques used to quantify accuracy, energy consumption, and CO2 emissions. 

Quantization is \textit{"the division of a quantity into a discrete number of small parts, often assumed to be integral multiples of a common quantify``}~\cite{DBLP:journals/corr/abs-2411-02530}. In LLMs, quantization is a compression technique that reduces the precision of model parameters from standard representation (e.g., 32-bit floating point) to lower-bit representation (e.g., 8-bit integers). Post-Training Quantization (PTQ) is preferred, primarily due to
the extensive computational demands associated with fine-tuning for Quantization Aware Training~\cite{zhang2023dual, DBLP:conf/icml/NagelABLB20, DBLP:journals/corr/abs-2006-10518}. In this work, we adopt PTQ as our compression method, thus adhering to the state-of-the-art in the community.

We use metrics for quantifying operational-level metrics. We adhere to international system measurements and community standards and quantify energy consumption in kilowatt-hour (kWh) and CO2 emissions in grams of CO2. To derive CO2 emission from energy consumption, we follow the vetted approach proposed by Niewenhuis et al. in Footprinter~\cite{DBLP:conf/wosp/NiewenhuisTIM24}; we present this formula in \Cref{eq:CO2-emissions}, where $C_e$ is the total amount of CO2 emissions, $C_i$ is the CO2 intensity, and $E$ is the total energy consumption.

\vspace*{-0.7cm}
\begin{align}
    C_{e} = C_{i} \times E
    \label{eq:CO2-emissions}
\end{align}
\vspace*{-0.6cm}

To quantify accuracy, we use Evaluate~\cite{DBLP:conf/emnlp/WerraTTLTPMRMN22}, a state-of-the-art tool for accuracy measurement, supporting various metrics; one such metric is BLUE, often used to evaluate machine-translations or F1 scores. Since different metrics are better suited to different uses of the LLM, we decided that using Evaluate’s metric collection interface would be the most efficient option. 
	
For the metrics, while this is subject to change, we have decided to use BLEU, F1 and BERT-score, and evaluating these various results to not only find which LLMs are the most accurate, but also in which categories they are most affected by the compression.


\section{Related Work}\label{sec:related}

\begin{table*}[ht]
    \centering
    \begin{tabular}{lllllll}
        \hline
        \textbf{Study} & \textbf {Quantization} & \textbf{Other technique} & \textbf{Energy Metrics} & \textbf{Accuracy} & \textbf{Resource Usage} & \textbf{CO2 Intensity}\\
        \hline
        \midrule
        Zhu et al. \cite{DBLP:journals/tacl/ZhuLLMW24} & \cmark & \cmark & \xmark & \xmark & \xmark & \xmark\\
        Afrin et al.\cite{DBLP:journals/corr/abs-2507-09665} & \cmark (AWQ) & \xmark & \xmark & \cmark (qual.) & \xmark & \xmark\\
        Khan et al.\cite{DBLP:journals/corr/abs-2504-06307} & \cmark & \xmark & \cmark & \cmark & \xmark & \xmark\\
        Agrawal et al.\cite{10968787} & \cmark & \cmark & \xmark & \cmark & \cmark & \xmark\\
        Yuan et al.\cite{DBLP:conf/cain/YuanSZCZSM24} & \xmark & \cmark (KD) & \cmark & \cmark & \cmark & \xmark\\
        \textbf{Our Work} & \cmark (PTQ) & \xmark & \cmark & \cmark & \cmark & \cmark\\
        \hline
        \bottomrule
    \end{tabular}
    \caption{Gap Analysis of Related Work on LLM Compression}
    \label{tab:gapanalysis}
\end{table*}

This section discusses scientific papers such as our experiment and their contributions, and how our study differs from or extends theirs. We present in \Cref{tab:gapanalysis} an overview and high-level comparison between the metrics of focus accross the analyzed papers.

Zhu et al. \cite{DBLP:journals/tacl/ZhuLLMW24} present a comprehensive review of LLM compression methods like quantization, pruning, knowledge distillation, and low-rank factorization. They elaborate on methods like Quantization-Aware Training (QAT) and Post-Training Quantization (PTQ) and propose benchmarking on FLOPs, inference time, and compression ratio. While their study aggregates existing methods, the present research presents an explicit empirical investigation of quantization, evaluating its overall effects on energy, resources, and accuracy.

Saima Afrin et al. \cite{DBLP:journals/corr/abs-2507-09665} conducted an empirical study examining the impact of Activation-aware Weight Quantization (AWQ) on the qualitative properties of automatically generated code by Large Code Models (LCMs), a custom sub-family of LLMs. In their work, employing cutting-edge static analysis tools, they observed that quantization has a tendency to preserve functional correctness of generated code and qualitative attributes like maintainability, readability, and structural complexity. They further examined the influence of model size on quality degradation following quantization, disproving the belief that loss of information from quantization damages such features. Afrin et al.'s study is especially focused on code quality as applied to Large Code Models, evaluating accuracy primarily in functional correctness and qualitative code properties. Our experiment broadens the perspective, checking the quantization's effect on a broader set of LLM uses (not just code creation) and quantitatively correlating it with measurably consumed energy and several resource consumption metrics, along with functional and qualitative correctness.

Khan et al. \cite{DBLP:journals/corr/abs-2504-06307} evaluate local inference and quantization for reducing the CO2 footprint of LLMs by as much as 45 percent less energy usage and minimal losses in accuracy. They concentrate on sustainability metrics, while we broaden the scope to also include memory, speed, and qualitative performance metrics.

Agrawal et al. \cite{10968787} investigate pruning, quantization, and distillation to deploy LLMs at the edge. They show quantization to be capable of reducing memory usage in half and lowering inference latency. In comparison to their multi-pronged strategy, our technique separates quantization to uncover its specific role in making efficiency-accuracy trade-offs.

Ye Yuan et al. \cite{DBLP:conf/cain/YuanSZCZSM24} estimated empirically the impact of \textbf{Knowledge Distillation} on the energy usage and efficiency of NLP models, such as BERT and GPT-2. They found that their distilled models consumed less energy, took less time to make inferences, and, in some cases, consumed less CPU and memory compared to non-distilled models. The work emphasized the importance of model selection to energy efficiency and performance, particularly for mobile and resource-limited applications. The central theme of Yuan et al.'s paper \cite{DBLP:conf/cain/YuanSZCZSM24} is knowledge distillation as the compression method. Our own experiment, however, specifically targets quantization compression of LLMs. we want to quantify its individual effects on energy consumption, different types of resource consumption measurement, and prediction performance, thus targeting a different but complementary compression setting.

Briefly speaking, prior work establishes the strengths of compression and optimization to efficiency and sustainability. Nevertheless, most survey many approaches, emphasize domain-specific tasks, or focus on other directions. Our work fills this gap by providing an end-to-end, empirical study of quantization's impact on LLMs on energy consumption, resource use, and accuracy
\newpage
\section{Experiment Definition}
% \begin{table*}[ht]
%     \centering
%     \begin{tabularx}{\textwidth}{@{}p{4.5cm} p{6cm} p{6cm}@{}}
%         \hline
%         \textbf{Goal} & \textbf{Research Questions} & \textbf{Metrics}\\
%         \hline
%         \midrule
%         Quantify the impact of quantization on server-hosted LLMs (Granite-7B, LLaMA-2 7B, Mistral-7B) with respect to energy, accuracy, efficiency, and CO2 emissions. & MRQ: How does quantization affect energy, accuracy, and resource utilization in LLMs? & Energy, CO2: kWh, J, gCO$_2$ = Ci × E
%         Time: latency (ms), throughput (tokens/s)\\
%         \hline
%          & RQ1: How to build a benchmarking system for baseline vs. quantized LLMs? & Resources: GPU\% / GPU memory (GB) / CPU\%\\
%          \hline
%          & RQ2: What are the accuracy–efficiency trade-offs? & Accuracy: MMLU (Acc\%)\\
%          \hline
%          & RQ3: Which model family benefits most from quantization? & Trade-offs: \% differences vs. baseline; accuracy–energy, latency–accuracy curves\\
%          \hline
%          & RQ4: How do CO2 emissions vary under different grid conditions? & favourable (off-peak, renewable), unfavourable (peak, fossil-heavy)\\
%          \hline
%          \bottomrule
%     \end{tabularx}
%     \caption{GQM Summary - Server-side Quantization of LLMs}
%     \label{tab:placeholder}
% \end{table*}

\begin{figure}[t]
    \centering
    \includegraphics[width=1.05\linewidth]{reportTemplate/figures/gqm.pdf}
    \caption{Goal-Question-Metric (GQM) framework overview this work follows. Stakeholders from multiple groups, especially students, LLM operators, and sustainability drivers could benefit from the software and the data produced in this work and released as open-science.}
    \label{fig:gqm-figure}
\end{figure}

% Report about the GQM (with figure).
In this section, we define the experiment and multi-layer analyze following the GQM-model for experimentation; \Cref{fig:gqm-figure} illustrates a high-level overview on the main goal, derived research questions, metrics employed, and various-groups stakeholders. Overall, in \Cref{sec:experiment:goal}, we identify the main goal of analyzing the impact of LLM compression by quatization on sustainability and performance metrics. To achieve these, we formulate and answer one main research question, and further divide into three sub-research questions; in \Cref{sec:experiment:questions}, we present the research questions that guide our experiments. Then, in \Cref{sec:experiment:metrics}, we present and expand on the metrics we use to quantify energy consumption, resource utilization, and accuracy.

\subsection{Goal}\label{sec:experiment:goal}
LLMs are widely adopted at the societal scale, yet raise performance and efficiency concerns, further cascading into sustainability concerns. Addressing this challenge, LLM researchers, engineers, and operators adopt quantization as a compression technique, which is expected to alleviate the efficiency concerns~\cite{zhang2023dual, DBLP:conf/icml/NagelABLB20}. However, it is still underexplored in the scientific community how quantization impacts performance, efficiency, and sustainability. 

We thus identify the main goal of 
\textit{(MG) evaluating the impact of quantization on operational-level LLM metrics.} Achieving the main goal, and, thus, answering the main research question (MRQ), would benefit stakeholders from various groups; for example, operations could gain deployment insights into compressed LLMs; sustainability operators would gain insights into the importance of CO2-aware workload scheduling against LLM ``optimization"\footnote{The quotation marks of this term satirize the invalidity, yet widely usage, of the term ``optimization". We argue that 1) since optimization means ``making something optimum", thus ``perfect", and 2) since nothing is either optimum or perfect in Computer Science, the term ``optimization" is invalid in our field.} techniques (e.g., compression by quantization; researchers and students can use the open-source benchmarking tool Quanti for exploration in the field.

We further identify the goal 
(G1) of designing and implementing a benchmarking system for quantifying LLMs, which would aid the experimentation process of identifying and evaluating impacts and tradeoffs.

We identify (G2) of evaluating how compression by quantization impacts operational-level metrics of LLMs, and hence comprehend how quantization can impact the real-world operation of these systems. 

Furthermore, we identify the goal 
(G3) of quantifying how accuracy impacts the energy consumption of LLMs, especially when LLMs differ in architecture (e.g., LLama2:7B from Meta underlies a different architecture than Granite:7B from IBM). While it is expected that higher accuracy requires higher computational power, and thus more energy consumption, it is yet unclear whether there is a relation, and to what extent, between accuracy and energy consumption. 

CO2-aware scheduling is widely employed in datacenter operation to meet Service Level Objectives and reduce the environmental footprint of massive-scale systems~\cite{DBLP:journals/corr/abs-2206-03259, nicolae5377101m3sa, DBLP:conf/wosp/NiewenhuisTIM24}. Following a similar approach, we identify the goal of evaluating CO2-aware scheduling of LLM inference; we thus explore, through (G4), the impact of running an LLM compressed by quantization, which is expected to consume fewer resources, yet at an unfavourable CO2-intensity timestamp, against running inference through an uncompressed LLM, which is expected to consume more resources, yet at a favourable CO2-intensity timestamp. Through this goal, we aim to analyze the importance of CO2-aware scheduling against compression techniques.


\subsection{Questions}\label{sec:experiment:questions}

We identify the main research question: \textit{\textbf{(MRQ)} How to explore the impact of compression by quantization of LLMs and how this technique impacts energy consumption, resource utilization, and precision?}

To answer MRQ, we identify four decomposed research questions, each corresponding to the goals identified in \ref{sec:experiment:goal}:

\textit{\textbf{(RQ1)} How to design and implement a benchmarking system for quantifying LLMs?} The essence of RQ1 involves designing and establishing the benchmarking tools for explainable and reproducible experimentation. To answer RQ1, also linked to G1, we propose Quanti, a benchmarking tool which aids in assessing, quantifying, and comparing the effects of LLMs quantization (\Cref{sec:design}).

\textit{\textbf{(RQ2)} How to evaluate the impact of compression by quantization on operational-level metrics of LLMs}. The essence of RQ2 involves examining the direct effects of quantization on sustainability and performance indicators of LLMs and quantify the magnitude of change when tranisioning from basedline to quantized model configurations. To answer RQ2, also linked to G2, we employ an engineered prototype of Quanti and measure energy consumption and inference time of four LLMs of identical size (i.e., same number of parameters), with and without quantization applied (\Cref{sec:experiments:operational-level}).

\textit{\textbf{(RQ3)} How to evaluate the tradeoff accuracy-energy consumption across various-architecture LLMs with and without compression by quantization?} The essence of RQ3 involves examining the effects of quantization on performance-level and sustainability-level indicators and identifying correlations. To answer RQ3, also linked to G3, we select four LLMs, of different architectures, and identify whether (and to what extent) arhitectural differences (e.g., LLaMA-2, Granite, Mistral) influence the trade-off characteristics (\Cref{sec:experiments:accuracy-energy}.

\textit{\textbf{(RQ4)} How to evaluate the impact of CO2 intensity for compressed and uncompressed models at favourable and unfavourable CO2 intensity day instances?} The essence of RQ4 is analysing (aimed at proving) the importance of workload scheduling CO2-awarely. To answer RQ4, also linked with G1, we select one LLM, a monthly CO2 intensity trace, and measure energy consumption. We compare the relative difference in CO2 emissions for running the same workload task on a baseline LLM at the CO2-best-overall CO2 intensity time interval of the day and on a quantised LLM at the CO2-worst-overall time interval (\Cref{sec:experiments:co2-aware}).

% \begin{itemize}
% \item \textbf{MRQ:} How to explore the impact of compression by quantization of LLMs and how this technique impacts energy consumption, resource utilization, and precision

% \item \textbf{RQ1:} How to design and implement a benchmarking system for quantifying LLMs?
% \item \textbf{RQ2:} How to evaluate the impact of compression by quantization of operational-level metrics of LLMs?

% \item \textbf{RQ3:} How to evaluate the tradeoff accuracy-energy consumption across various-architecture LLMs under quantization and without quantization?

% \item \textbf{RQ4:} How to evaluate the impact of CO2 intensity for compressed and uncompressed models at favourable and unfavourable CO2 intensity day instances?


% \end{itemize}

%  - Our main question comprises of how we assess the impact of quantization on LLMs (e.g LLama2 7B, Granite 7B, and Mistral 7B) and how it effects the energy usage (kWh, Joules), inference time (ms), accuracy (percent) and resource utilization (CPU/GPU load percent or memory in GB) when deployed on server based system. 

% sub questions:

% - to design a benchmarking system that supports systematic evaluation of energy usage (kWh), inference latency (ms), resource utilization (CPU/GPU percent and memory GB) and accuracy trade-offs

% - How can we identify and quantify the trade-offs between accuracy and efficiency when comparing baseline and quantized models specifically in terms of reduced energy use (kWh, Joules) and memory footprint (GB) versus possible accuracy drops in percent.
%  for example, quantization can reduce energy usage and may be memory, but it can drop the accuracy. so the objective of this question is to understand the sweet spot between sustainability and performace.

% - 


% - to identity which llm model shows benefits (in terms of sustainability and perfomance) from quantization. so the idea is to see wether Granite, LLaMA or Mistral will get the energy savings, effiency with accuracy, as not all models will be same to get these benefits.

% - Similarly, to evaluate the CO2 intesity of baseline and quantized models at different time periods of the day (favourable and unfavourable CO2 intensity day instances). for example, running the same model at peak grid load vs off-peak renewable heavy hours, may produce differnt CO2 footprints. so we can see how quantization interacts with real world energy sourcing .

\subsection{Metrics}\label{sec:experiment:metrics}
To answer the research questions, the experiment will record both operation-level and task-level quality metrics so each is measured with explicit units of measurement for reproducibility.

We will measure energy usage in kilowatt-hours (kWh) and Joules through utilities such as pyJoules, CodeCarbon, or Experiment Runner. This allows us to log the total energy required to execute inference tasks and to scale energy per request or per token.

Inference latency will be tracked along two dimensions: per-prompt latency (milliseconds), measuring responsiveness, and throughput in tokens per second, measuring scalability.

Resource utilization will include GPU utilization (percent), GPU memory utilization (GB), and CPU usage (percent) during inference. These will reflect the performance relative to efficiency in utilization of computational resources when models are quantized versus their baseline equivalents.

Accuracy will be measured with a number of benchmarks, each having domain-agnostic metrics:

- MMLU: Knowledge and reasoning task accuracy (\%).

- TriviaQA: F1 score (percent) and Exact Match (EM) on question answering.

- CNN/DailyMail or XSum: ROUGE and BLEU on summarization.

The intensity of CO2 will be measured by transposing energy consumption into equivalent CO$_2$ emissions, per the formula Ce = Ci × E, where Ce is the total emissions (grams CO$_2$), Ci is the grid CO2 intensity (gCO$_2$/kWh), and E is energy consumed (kWh). For the sake of realism, this will be estimated for optimal (renewable-abundant, off-peak) and extreme (fossil-fuel abundant, peak) grid conditions.

Finally, relative trade-offs among similar levels of performance will be compared by relative percentage differences between baseline and quantized counterparts across all metrics. Accuracy–energy trade-off and latency–accuracy curves will be plotted as well to illustrate sustainability vs. performance trade-offs.






\input{reportTemplate/sections/planning}
\section{Design of Quanti}\label{sec:design}

\subsection{Requirement Analysis}

\subsection{Design choices}

\textcolor{red}{Page limit: 1}
 s
\section{Experiment Execution}\label{sec:experiments}
Report about: how you plan to conduct your experiment, which tools you are going to use, which devices/laptops, figure and description of the overall software/hardware infrastructure you are setting up for the experiment (\eg who communicates with whom, proxies, network requests, order of actions, \etc). 


\subsection{Evaluating quantization impact on operational-level metrics}\label{sec:experiments:operational-level} \label{sec:exp1}

\subsection{Evaluating accuracy-energy tradeoff for various-architectural LLMs, with and without compression}\label{sec:experiments:accuracy-energy} \label{sec:exp2}

\subsection{Evaluating CO2-aware scheduling against compression techniques}\label{sec:experiments:co2-aware} \label{sec:exp3}


\textcolor{red}{Page limit: 2}


\input{reportTemplate/sections/results}
\input{reportTemplate/sections/discussion}
\input{reportTemplate/sections/threats}
\input{reportTemplate/sections/conclusion} 

\begin{figure*}
    \centering
    \includegraphics[width=0.95\linewidth]{reportTemplate/figures/paper-page-budgeting-green-lab.pdf}
    \caption{Page Budgeting following the template suggested in the work in progress paper by Iosup et al. in \cite{iosup-epema-systems-writing-wip}. Edit link: \url{https://drive.google.com/file/d/1-9jcqbOQJRCJZpWZldTrm27HNUFuAtWR/view?usp=drive_link}.}
    \label{fig:placeholder}
\end{figure*}

\bibliographystyle{IEEEtran}
\bibliography{references}

\end{document}
